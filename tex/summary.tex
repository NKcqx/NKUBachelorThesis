% !TeX root = ../main.tex
% -*- coding: utf-8 -*-
\chapter{总结与展望}
实现本篇论文的工作主要分为三部分:首先是调研了相关文献来了解链接预测的背景、经典算法以及近些年提出的较有代表性的算法,并在之后提出了自己的算法;
之后设计了一系列实验来验证算法的性能,并根据实验现象对算法做了调整,最终得到了完整的算法模型及其在实际问题中的应用效果;
最后用一些实验从数据的角度上探究了影响算法性能的几个因素。


未来的工作也许可以从以下几点入手:
第一,在研究网络聚集程度对算法性能的影响中,实验并不充分,得出“网络聚集程度越高,算法准确率越高”的结论还需要扩大对数据集聚类系数的选取范围,并增大数据量;
第二,实验中未考虑数据集间噪声的不平衡性,日后可以从添加噪声和降噪这两个角度来平衡数据集间的噪声差异;
第三,在所有评估算法性能的实验中,均为提及运行时间这一指标。事实上,受算法复杂性影响,CLML需要很长的运行时间,这一点和其它算法对比时是一个劣势,今后也许可以对运行效率做进一步优化。