% !TeX root = ../main.tex
% -*- coding: utf-8 -*-


\begin{zhaiyao}

近年来,网络数据量已呈爆炸式增长,人们每天都产生着海量的数据。如何有效利用这些数据,
从中挖掘出有价值的信息帮助人们认识世界成为了一个重要课题,链接预测问题也由此产生。


链接预测可以挖掘出网络中丢失的链接或进行网络动态分析,是机器学习领域中的一个重要课题。
随着相关研究的深入,人们对算法的需求愈加复杂,例如需要算法有对层次结构数据或带有辅助信息数据的处理能力,或是能使用序列化输入的方式训练模型。
本文的研究目的之一就是如何有效利用辅助信息以提高模型的预测精度。
由于现实数据的维度往往很高,直接使用时模型的训练成本较高,
考虑到数据通常为嵌入在高维空间的低维流形结构,本文首先提出了一个基于线性流形学习的算法来解决高维数据的问题。
在此基础上,本文又利用辅助信息,使用同样地方式构建了一个双流形结构,利用两个流形结构的一致性提出了一个双边协同学习算法。
算法通过协同学习的方式令两个流型相互补全,成功利用辅助信息解决了由数据稀疏带来的流型结构不完整的问题。


此外,虽然近年来涌现了众多优秀的算法,但一个很常见的问题就是与算法相关的实验较匮乏,
因此,本文的另一个研究目的就是从不同角度设计实验来检验算法性能和分析影响因素。
为了检验算法性能,本文对算法做了交叉验证、鲁棒性检验、后续检验,并最终发现所提算法与其它算法相比,性能有显著提升;
为了分析影响算法性能的因素,文中又从数据的结构上出发,设计了多个实验并做了解释分析;
为了验证算法在实际应用中的有效性,文中最后将该算法应用到了对药物间关联的预测中并得到了令人满意的结果。
\end{zhaiyao}


\vspace{20pt}

\begin{guanjianci}
链接预测;流形学习;协同学习;实验设计
\end{guanjianci}



\begin{abstract}


In recent years, the amount of data in network has exploded and people can generate massive amounts of data every day. 
How to utilize those data, digging out some useful information buried underground to give us a hint of real world, 
has become an important subject and thus generating a problem called link prediction.


Link prediction can be used to extract missing information, identify spurious interactions, evaluate network evolving mechanisms, and so on. 
So that it can be used in recommendation systems, Pathogenic gene mining and drug interaction mining. 
In this article, a linear manifold learning algorithm has been devised to uncover novel interactions on a global scale 
since data are usually embedded in low-dimensional manifolds. Then a collaborative learning method has been introduced 
with the help of auxiliary information to solve the problem of data sparsity.


To evaluate the overall performance of our method and find out some relationships between network topology structure 
and predicting results, several experiments have been carried out. In the cross validation experiments, 
our method achieved better results than eight other state-of-the-art methods in most cases and has better robustness. 
Finally, we loaded some real DDI data to test our method and found that it predicted the left-out interactions reasonably well.

\end{abstract}


\vspace{20pt}


\begin{keywords}
link prediction; manifold learning; collaborative learning; experimental design and analysis
\end{keywords} 